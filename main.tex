%%%%%%%%%%%%%%%%%%%%%%%%%%%%%%%%%%%%%%%%%%%%%%%%%%%%%%%%%%%
%% Congratulations, you've made an excellent choice
%% of writing your Tampere University dissertation using
%% the LaTeX system. This document attempts to be
%% as complete a template as possible to let you focus
%% on the most important part: the writing itself.
%% Thus the details regarding the visual appearance
%% have already been worked out for you!
%%
%% I sincerely hope you will find this template useful
%% in completing your dissertation project. I've tried to
%% add comments (followed by the % sign) to clarify
%% the structure and purpose of some of the commands.
%% Most of the magic happens in the file
%% taudissertation.cls, which you are more than welcome
%% to take a look at. Just refrain from editing it
%% in the most crucial versions of the dissertation!
%%
%% I wish you and your dissertation project the best
%% of luck! If this template causes you trouble along
%% the way or if you've any suggestions for improving it,
%% please be in contact through GitHub.
%% (https://github.com/villekol/tau-latex-dissertation-template)
%%
%% Yours,
%%
%% Ville Koljonen
%%
%% PS. This template or its associated class file don't
%% come with a warranty. The content is provided as is,
%% without even the implied promise of fitness to the
%% mentioned purpose. You, as the author of the dissertation,
%% are responsible for the entire work, including the
%% provided material. No one else is liable to you for
%% any damage inflicted on you or your thesis, were it
%% caused by using this template or not.
%%%%%%%%%%%%%%%%%%%%%%%%%%%%%%%%%%%%%%%%%%%%%%%%%%%%%%%%%%%

%%%%% INSTRUCTIONS FOR COMPILING THE DOCUMENT %%%%%
%% Overleaf: just click Recompile.
%% Terminal:
%%  1. pdflatex main.tex
%%  2. makeindex -s main.ist -t main.glg -o main.gls main.glo
%%  3. biber main
%%  4. pdflatex main.tex
%%  5. pdflatex main.tex
%% Similar sequence of commands is also required
%% in LaTeX specific editors.
%%%%%%%%%%%%%%%%%%%%%%%%%%%%%%%%%%%%%%%%%%%%%%%%%%%

%%%%% METADATA %%%%%
%% Always keep the following metadata up to date!
%% This is important for your PDF file to comply
%% to accessibility standards.
%% (And yes, this information must remain here,
%% before \documentclass[...]{...}.)

% \Title and \Language are mandatory,
% others desirable
% The appropriate Finnish language code is 'fi',
% UK English is en-UK
\begin{filecontents*}[overwrite]{\jobname.xmpdata}
\Title{Title}
\Author{Firstname Lastname}
\Language{en-US}
\end{filecontents*}

\pdfminorversion=6

%%%%% PREAMBLE %%%%%

%%%%% Document class declaration.
% The possible optional arguments are
%   finnish - thesis in Finnish (default)
%   english - thesis in English
%   numeric - citations in numeric style (default)
%   authoryear - citations in author-year style
%   apa - citations in APA 7 style (English only)
%   ieee - citations in IEEE style (English only)
% Example: \documentclass[english, authoryear]{tauthesis}
%          thesis in English with author-year citations
\documentclass[english]{taudissertation}

% The glossaries package may throw a warning:
% No language module detected for 'finnish'.
% You can safely ignore this. All other
% warnings should be taken care of!

%%%%% Your packages.
% Before adding packages, see if they can be found
% in taudissertation.cls already. If you're not sure
% that you need a certain package, don't include it
% in the document! This can dramatically reduce
% compilation time.

\usepackage{lipsum}

%\usepackage{amsmath,amssymb}

% (SI) unit handling
%\usepackage{siunitx}
%
%\sisetup{
%    detect-all,
%    exponent-product=\cdot
%}

%\if@langenglish
%    \sisetup{output-decimal-marker={.}}
%\else
%    \sisetup{output-decimal-marker={,}}
%\fi

%%%%% Your commands.

% Print verbatim LaTeX commands
\newcommand{\verbcommand}[1]{\texttt{\textbackslash #1}}

%%%%% Glossary information.

% Use the following lines ONLY if you need more
% than one glossary. The first argument specifies
% a type label for the glossary and the second
% the displayed name.
% \newglossary*{symbs}{Symbols}
% \newglossary{label}{Displayed name}
% ...

\makeglossaries

% Use this line if using the default glossary.
% Otherwise comment out.
\loadglsentries[main]{tex/glossary.tex}

% Use this line if using more than one glossary.
% Otherwise comment out.
% \loadglsentries[symbs]{tex/glossary2.tex}

%%%%% Citation information.

% Use a separate .bib file to store bibliographic
% information relating to own publications.
% If you're writing a monograph, simply comment
% this out.
\addbibresource{publications/publications.bib}

% References for the dissertation.
\addbibresource{tex/references.bib}

\hypersetup{hidelinks}

\begin{document}

%%%%% FRONT MATTER %%%%%

\frontmatter

% Enter the title and a possible subtitle
% of your work along with your name here.
\title{Title}
\subtitle{Subtitle}
\author{Firstname Lastname}

% Notice that this is only included for
% convenience, use the separate title page
% template to produce the final book cover pages.
\maketitle

% Write the dedication into a separate file.
% Comment out if not applicable.
\dedication{tex/dedication.tex}

% Write either 'Preface' ('Esipuhe') or
% 'Acknowledgements' ('Kiitokset').
% Write the title of this section as the first
% argument and the file name as the second.
\preface{Preface/Acknowledgements}{tex/preface.tex}

% Write the abstracts, the first one in the
% main language of the dissertation.
% Comment the other one out if not applicable
% (e.g. international students).
\abstract{tex/abstract.tex}
\otherabstract{tex/otherabstract.tex}

% Automatically updating ToC.
\tableofcontents

% Print lists of figures, tables and programs,
% if necessary. Otherwise comment out.
% Use the listings package for all code listings.
\listoffigures
\listoftables
\lstlistoflistings

% Misc stuff related to how the glossary is displayed.
% You can especially tweak the lengths to suit you!
\glsaddall
\setglossarystyle{taulong}
\setlength{\glsnamewidth}{0.25\textwidth}
\setlength{\glsdescwidth}{0.75\textwidth}
\renewcommand*{\glsgroupskip}{}

% Print the default glossary of abbreviations,
% if necessary. Otherwise comment out.
% The appropriate Finnish variant is 'Lyhenteet'
\printglossary[title=Abbreviations]

% Print more than one glossary with these lines.
% Otherwise comment out.
% \printglossary[type=symbs]
% \printglossary[type=label]
% ...

% Print the list of original publications.
% The format of the bibliographic information
% is similar to the entries in the bibliography.
% Comment out if writing a monograph.
\listofpublications
% \clearpage
\authorcontribution{tex/authorcontribution.tex}

%%%%% MAIN MATTER %%%%%

\mainmatter

% Write each chapter into a separate file.
% Add a \label right after the chapter name
% if you need to reference it later.

\chapter{Using this template}
\label{ch:firstchapter}
This dissertation template aims to be as complete a guide as possible for writing a Tampere University PhD theses. For the most up to date instructions, visit the \href{https://libguides.tuni.fi/dissertationpublishing}{library webpage} for dissertations. The purpose of the present short introduction is to familiarize you to some of the less obvious \LaTeX-stuff that the official instructions do not cover. We do, however, assume that you are already familiar with the \LaTeX{} system and have written something using it.

\section{Structure of the template}

The template is an example of a multifile \LaTeX{} project, where there is a single master file (\texttt{main.tex}). This file is responsible for including all the relevant content to the final PDF, which is the result of running \texttt{pdflatex} and other commands on it. The files you mostly need to worry about should be put into the directory \texttt{/tex/}, and they include one content file for each of the following:
\begin{itemize}
\item dedication (if used),
\item preface/acknowledgments (acknowledgements in UK English),
\item abstracts (main language and possibly your own language),
\item abbreviations*,
\item chapters in the main body of the work,
\item bibliography*,
\item appendices (if used).
\end{itemize}
Here * signifies a specially formatted file, others are simple \LaTeX{} text files.

In addition to the above the template contains a directory (\texttt{/img/}) where you could put anny image files to include in the work, a template version history file (not always up to date), the class file \texttt{taudissertation.cls} that defines the styles used in the document, and most importantly a directory called \texttt{/publications/}. This is another important place to consider when using this template to write most dissertations. If you are writing a monograph, you can just ignore or even delete it. Otherwise this directory is meant to contain
\begin{itemize}
\item the PDF files of your own publications included in the dissertation,
\item a separate bibliography file for specifying their bibliographic information.
\end{itemize}
We will come back to how this directory functions later when considering the relevant parts of your dissertation.

In addition to the supplied file directory structure, this template specifies certain parts that your dissertation should contain. The structure of the \texttt{main.tex} file reflects this division, and can be described from top to bottom as follows.
\begin{enumerate}
\item The \emph{pre-}preamble. Normally, a \LaTeX{} source file begins with the command \texttt{\textbackslash documentclass[\ldots]\{\ldots\}}. This template adds commented instructions \emph{before} this and more crucially a mechanism for specifying PDF document metadata there.
\item The preamble. Everything that comes between \texttt{\textbackslash documentclass[\ldots]\{\ldots\}} and \texttt{\textbackslash begin\{document\}}. Consists of mostly template-related stuff you can modify for your own needs, but also the place for you to include your own packages or define your own commands.
\item The front matter. You are essentially writing a book, and it is customary to put everything between title pages and content descriptions in a separate part where the pages have Roman numbering. In a dissertation this part also contains information aboout your own publications (except in monographs).
\item The main matter. Basically everything else from chapters to appendices. Writing this part means writing regular \LaTeX.
\item The ``publication matter''. A special part of dissertations, where you put your own publications. Adding them from external PDF files along with cover pages is mostly automated in this template. (Obviously not needed in monographs.)
\end{enumerate}
Equipped with this information you should be able to understand and use the template easily. Following the example set by the \texttt{main.tex} file is a good idea for starting the work.

\section{Styles in the template}

The shortest possible description of these is that everything has been set up to match the official Word template for Tampere University dissertations as closely as possible in the \LaTeX{} world. Therefore there should be no need to change anything. If something does, however, seem odd or wrong, please contact the maintainer (contact information at the top of \texttt{taudissertation.cls}) if you need help sorting it out, or if it is something that should be updated in the template.

Some of the packages that are used in the template (especially regarding accessibility of the PDF document) are relatively new to the most common \LaTeX{} distributions. The template has been tested to work as expected in the Overleaf environment with the \TeX{} Live 2020 compiler. If you encounter issues compiling the basic template on a local system, please check that you
\begin{enumerate}
\item have an up-to-date \LaTeX{} distribution,
\item are using the correct commands for compiling (see top of \texttt{main.tex}).
\end{enumerate}

\section{Accessibility considerations}

The Finnish law, following the European Union directive 2016/2102 on accessibility of the websites and mobile applications of public sector bodies, demands that any electronic publications are made appropriately accessible (within reason). This also applies to your dissertation, and therefore you should know or find out what is expected of you and your work. The writer or maintainer of the template assumes no responsibility on behalf of dissertation authors!

This template aims to help as much as possible with improving the accessibility of the final document. Unfortunately, as of 2021 (and most likely up to 2024), the \LaTeX{} system is incapable of producing PDF files that would conform to the accessible PDF/UA standard, or even the slightly less demanding university guidelines. The main issue is that \LaTeX{} carefully throws away much of the needed structural information as soon as possible to save memory (which was a real issue back in the 80's and 90's). This does not mean that all is lost, and you should still strive for maximum accessibility!

As with the required styles in dissertations, this template tries to take care of many things automatically for you. The few things you do need to take into account while writing are
\begin{enumerate}
\item the language you use is clear and unambiguous, and that nothing essential is conveyed through visual formatting only,
\item alternative texts (or alt texts) for \emph{all} images you include in the thesis,
\item required document metadata that include the title and the main language of your work,
\item compatibility of mathematics environment with the automated alt text solution.
\end{enumerate}
The instructions for addressing these are as follows.
\begin{enumerate}
\item Do as was described. Use external services for assessing the easy-to-read nature of your text, if necessary.
\item Use the command \texttt{\textbackslash pdftooltip\{\ldots\}\{\ldots\}} from the \texttt{pdfcomment} package (included automatically).
\begin{verbatim}
\begin{figure}
\pdftooltip{\includegraphics[...]{...}}%
{Figure. This alternative text describes
what seeing users see in the image.}
\caption{Figure caption}
\label{fig:somelabel}
\end{figure}
\end{verbatim}
Notice that the caption and the alternative text of an image serve entirely different purposes, and there for the alt text should never be just a copy of the caption! (Also screen readers will always read the caption in addition to the alt text.) Be verbose in the alt texts, but try to focus on the most essential meanings conveyed by the image.
\item Always keep the document metadata fields in the pre-preamble of \texttt{main.tex} up-to-date.
\item Never use the old \TeX{} style \verb+$...$+ and \verb+$$...$$+ environments for math, and instead use the \LaTeX{} style \verb+\(...\)+ and \verb+\[...\]+. The double dollar display math should never be used anyway.

Most of the math environments from \texttt{amsmath}, such as \texttt{equation}, \texttt{equation*} and \texttt{align}, are fine to use and get correct alt texts automatically. In contrast, no math environment from a package other than \texttt{amsmath} is supported! The alternative text for all mathematics environments is the \LaTeX{} source code, until a better solution arrives.
\end{enumerate}

\subsection{Subsection}

Finally, here is an example of a subsection with some float examples to fill the table of contents.

\begin{figure}
    \centering
    \caption{Figure caption}
\end{figure}

\begin{table}
    \centering
    \caption{Table caption}
\end{table}

\renewcommand{\lstlistingname}{Program}
\lstinputlisting
    [float,
    caption={Program caption},
    label=prog:esimerkki,
    language=C,
    numbers=left]
    {tex/example.c}

\subsubsection{Subsubsection}

And finally also a subsubsection where we put an example citation \cite{somecitation}. The only necessary command with numeric citation styles (numeric, ieee) is \verb+\cite{...}+, but with author-year styles (authoryear, apa) you'll want to test at least \verb+\parencite{...}+, \verb+\parencite*{...}+, \verb+\textcite{...}+ and \verb+\citeauthor{...}+. The last could also be useful for numeric styles on some occasions.


% Add other chapters similary.

% Print the list of references.
%\setlength{\labelnumberwidth}{3em}
\printbibliography[heading=bibintoc, notkeyword={thisdissertation}]

% Write the appendices as chapters between
% \begin{appendices} and \end{appendices}.

\begin{appendices}

\chapter{Appendix}
\label{ch:appendix}
\lipsum[1]

\end{appendices}

%%%%% PUBLICATION MATTER %%%%%
% Comment this part out completely,
% if writing a monograph.

\publicationmatter

% Include the publications using the command
%
% \publication{<citeID>}{<path/to/article/pdf>}
%
% This automatically prints the publication
% cover pages and appends the article pdf
% to this dissertation. Remember to include
% them in the same order as in the list of
% publications!

\publication{article1}{publications/article-1.pdf}
\publication{article2}{publications/article-2.pdf}
\publication{article3}{publications/article-3.pdf}

\end{document}